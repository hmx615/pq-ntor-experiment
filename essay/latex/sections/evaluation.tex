% Section 5: Evaluation
% 【中文翻译】第五章:实验评估
% Converted from Markdown to LaTeX
% Date: 2025-11-27

\section{Evaluation}
\label{sec:evaluation}
% 【中文翻译】第五章 实验评估

We conduct a comprehensive three-phase evaluation to assess the performance and feasibility of \pqntor\ in space-air-ground integrated networks (\sagin).
% 【中文翻译】我们进行了全面的三阶段评估,以评估Hybrid-NTOR在空天地一体化网络(SAGIN)中的性能和可行性。

\subsection{Experimental Setup}
\label{sec:eval:setup}
% 【中文翻译】5.1 实验设置

Our experiments use a heterogeneous hardware testbed spanning both x86\_64 and ARM64 architectures, summarized in Table~\ref{tab:hardware}.
% 【中文翻译】我们的实验使用了跨越x86_64和ARM64架构的异构硬件测试平台,如表1所示。

\subsubsection{Hardware Configuration}
% 【中文翻译】硬件配置

\begin{table}[t]
\centering
\caption{Hardware Configuration}
% 【中文翻译】硬件配置
\label{tab:hardware}
\small
\begin{tabular}{@{}llll@{}}
\toprule
\textbf{Device} & \textbf{CPU} & \textbf{Arch} & \textbf{RAM} \\
\midrule
Dev Machine & Intel i9-14900 & x86\_64 & 32 GB \\
Phytium Pi $\times$6 & FTC664 @ 2.3GHz & ARM64 & 8 GB \\
\bottomrule
\end{tabular}
\end{table}

\textbf{Platform Comparison:}
% 【中文翻译】平台对比:
\begin{itemize}[leftmargin=*,noitemsep]
    \item \textbf{x86\_64 (WSL2):} Used for Phase 1 micro-benchmarks and initial development
    % 【中文翻译】x86_64(WSL2):用于第一阶段微基准测试和初始开发
    \item \textbf{ARM64 (Phytium Pi):} Used for Phase 3 distributed deployment, validating real-world applicability on resource-constrained embedded platforms
    % 【中文翻译】ARM64(飞腾派):用于第三阶段分布式部署,验证在资源受限嵌入式平台上的实际应用性
\end{itemize}

\subsubsection{Software Stack}
% 【中文翻译】软件栈

Table~\ref{tab:software} summarizes the complete software stack used in our implementation and experiments.
% 【中文翻译】表格总结了我们实现和实验中使用的完整软件栈。

\begin{table}[t]
\centering
\caption{Software Components}
% 【中文翻译】软件组件
\label{tab:software}
\small
\begin{tabular}{@{}lll@{}}
\toprule
\textbf{Component} & \textbf{Version} & \textbf{Purpose} \\
\midrule
Hybrid-Tor Core & Custom C & Complete Hybrid-NTOR handshake \\
liboqs & 0.11.0 & Kyber-512 KEM operations \\
OpenSSL & 3.0.2+ & HKDF, HMAC, SHA-256 \\
GCC & 11.4.0 & C compiler with -O2 \\
Python & 3.10+ & Test automation, analysis \\
tc/netem & Kernel & Network delay simulation \\
Skyfield & 1.48 & Satellite orbit calculation \\
Flask & 2.3.0 & Web dashboard backend \\
\bottomrule
\end{tabular}
\end{table}

\textbf{Key Implementation Details:}
% 【中文翻译】关键实现细节:
\begin{itemize}[leftmargin=*,noitemsep]
    \item Compiler flags: \texttt{-O2 -Wall -Wextra -std=c11}
    % 【中文翻译】编译器标志:-O2 -Wall -Wextra -std=c11
    \item liboqs configuration: Kyber-512 (NIST Level 1, equivalent to AES-128)
    % 【中文翻译】liboqs配置:Kyber-512(NIST一级,等效于AES-128)
    \item Time measurement precision: Microsecond (\us) using \texttt{gettimeofday()}
    % 【中文翻译】时间测量精度:微秒级,使用gettimeofday()
\end{itemize}

\subsubsection{Network Topologies}
% 【中文翻译】网络拓扑

Our evaluation spans \textbf{12 distinct network topologies} designed to represent diverse \sagin\ scenarios, ranging from pure terrestrial networks to complex multi-tier space-air-ground architectures.
% 【中文翻译】我们的评估涵盖了12种不同的网络拓扑,旨在代表多样化的SAGIN场景,从纯地面网络到复杂的多层空天地架构。

\paragraph{Topology Categories}
% 【中文翻译】拓扑分类

We categorize the 12 topologies into four groups based on network characteristics, as shown in Table~\ref{tab:topo-categories}.
% 【中文翻译】我们根据网络特性将12种拓扑分为四组,如表所示。

\begin{table}[t]
\centering
\caption{Topology Categories Overview}
% 【中文翻译】拓扑分类概述
\label{tab:topo-categories}
\small
\begin{tabular}{@{}llp{4cm}@{}}
\toprule
\textbf{Category} & \textbf{IDs} & \textbf{Description} \\
\midrule
Pure NOMA & T01-T02 & Terrestrial NOMA with direct satellite uplink \\
Single-Tier Space & T03-T06 & LEO/MEO satellite integration \\
Multi-Hop SAGIN & T07-T09 & Space + Air + Ground hybrid \\
Complex Hybrid & T10-T12 & Multi-tier cooperative networks \\
\bottomrule
\end{tabular}
\end{table}

Table~\ref{tab:topologies} provides detailed specifications for all 12 topologies.
% 【中文翻译】表格提供了所有12种拓扑的详细规格。

\begin{table*}[t]
\centering
\caption{Detailed Topology Specifications}
% 【中文翻译】详细拓扑规格
\label{tab:topologies}
\footnotesize
\begin{tabular}{@{}lllrrrrr@{}}
\toprule
\textbf{ID} & \textbf{Name} & \textbf{Hops} & \textbf{Node Types} & \textbf{Delay (ms)} & \textbf{BW (Mbps)} & \textbf{Loss (\%)} & \textbf{NOMA} \\
\midrule
T01 & Z1 Up-1 Direct & 2 & UAV + SAT & 20 & 50 & 0.5 & \checkmark \\
T02 & Z1 Up-2 Multi-NOMA & 3 & 2×UAV + SAT & 35 & 30 & 1.0 & \checkmark \\
T03 & Z2 LEO Single & 2 & Terminal + LEO & 40 & 25 & 1.5 & \\
T04 & Z3 LEO Multi & 3 & 2×Term + LEO & 60 & 20 & 2.0 & \\
T05 & Z5 MEO Relay & 3 & UAV + MEO + Ground & 90 & 15 & 2.5 & \checkmark \\
T06 & Z6 GEO Hybrid & 4 & 2×UAV + GEO & 120 & 10 & 3.0 & \checkmark \\
T07 & Z1 Down Multi & 4 & SAT + 2×UAV + Term & 80 & 20 & 2.0 & \checkmark \\
T08 & Z2 Air-Ground & 3 & UAV + Ground + SAT & 70 & 25 & 1.5 & \\
T09 & Z3 Multi-Tier & 4 & LEO + MEO + UAV & 100 & 18 & 2.5 & \\
T10 & Z4 Cooperative & 3 & 2×SAT + Ground & 85 & 22 & 2.0 & \checkmark \\
T11 & Z5 Complex & 4 & LEO + UAV + 2×Ground & 95 & 20 & 2.2 & \checkmark \\
T12 & Z6 Full SAGIN & 5 & GEO + MEO + LEO + UAV & 150 & 12 & 3.5 & \checkmark \\
\bottomrule
\end{tabular}
\end{table*}

\paragraph{Link Delay Simulation}
% 【中文翻译】链路延迟模拟

We use Linux \texttt{tc} (traffic control) with \texttt{netem} (network emulation) to simulate realistic \sagin\ link characteristics. Example configurations:
% 【中文翻译】我们使用Linux tc(流量控制)配合netem(网络仿真)来模拟真实的SAGIN链路特性。配置示例:

\begin{lstlisting}[language=bash,caption={Network Delay Simulation Examples}]
# LEO satellite link (800 km altitude)
tc qdisc add dev veth0 root netem delay 10ms 2ms loss 0.5%

# GEO satellite link (35,786 km altitude)
tc qdisc add dev veth1 root netem delay 250ms 10ms loss 1.0%

# UAV-to-ground link with jitter
tc qdisc add dev veth2 root netem delay 5ms 1ms loss 0.1%
\end{lstlisting}

\paragraph{Satellite Link Parameters}
% 【中文翻译】卫星链路参数

Based on propagation delay: RTT = 2 × distance / c, we use the parameters shown in Table~\ref{tab:satellite-params}.
% 【中文翻译】基于传播延迟:RTT = 2 × 距离 / c,我们使用表中所示的参数。

\begin{table}[t]
\centering
\caption{Satellite Link Parameters}
% 【中文翻译】卫星链路参数
\label{tab:satellite-params}
\footnotesize
\begin{tabular}{@{}lrrrl@{}}
\toprule
\textbf{Orbit} & \textbf{Altitude} & \textbf{1-Way} & \textbf{RTT} & \textbf{Example} \\
\midrule
LEO & 500-2,000 km & 1.7-6.7 ms & 3.3-13.3 ms & Starlink \\
MEO & 8,000-20,000 km & 27-67 ms & 53-133 ms & GPS, O3b \\
GEO & 35,786 km & 119 ms & 238 ms & Intelsat \\
\bottomrule
\end{tabular}
\end{table}

\subsubsection{Performance Metrics}
% 【中文翻译】性能指标

We define the following metrics across all experimental phases:
% 【中文翻译】我们定义了以下跨所有实验阶段的指标:

\paragraph{Phase 1: Handshake Performance Metrics}
% 【中文翻译】第一阶段:握手性能指标

\begin{itemize}[leftmargin=*,noitemsep]
    \item \textbf{Full Handshake Latency} (\us): End-to-end time from \texttt{client\_create\_onionskin()} to \texttt{client\_finish\_handshake()}
    % 【中文翻译】完整握手延迟(微秒):从client_create_onionskin()到client_finish_handshake()的端到端时间
    \begin{itemize}[noitemsep]
        \item Includes: Kyber-512 KEM + HKDF key derivation + HMAC authentication
        % 【中文翻译】包含:Kyber-512 KEM + HKDF密钥派生 + HMAC认证
        \item Statistics: Min, Median, Average, Max, StdDev
        % 【中文翻译】统计指标:最小值、中位数、平均值、最大值、标准差
        \item Sample size: 1000 iterations (with 10 warm-up)
        % 【中文翻译】样本大小:1000次迭代(含10次预热)
    \end{itemize}

    \item \textbf{Component Breakdown} (\us):
    % 【中文翻译】组件分解(微秒):
    \begin{itemize}[noitemsep]
        \item Client Create: Generate Kyber keypair and create onionskin
        % 【中文翻译】客户端创建:生成Kyber密钥对并创建洋葱皮
        \item Server Reply: KEM encapsulation and generate reply
        % 【中文翻译】服务端响应:KEM封装并生成响应
        \item Client Finish: KEM decapsulation and verify authentication
        % 【中文翻译】客户端完成:KEM解封装并验证认证
    \end{itemize}

    \item \textbf{Throughput} (handshakes/sec): $1 / \text{avg\_full\_handshake\_latency}$
    % 【中文翻译】吞吐量(握手次数/秒):1 / 平均完整握手延迟
\end{itemize}

\paragraph{Phase 2 \& 3: Network Performance Metrics}
% 【中文翻译】第二和第三阶段:网络性能指标

\begin{itemize}[leftmargin=*,noitemsep]
    \item \textbf{Circuit Build Time (CBT)} (\ms): Time to establish a 3-hop Tor circuit
    % 【中文翻译】电路构建时间(CBT,毫秒):建立三跳Tor电路的时间
    \item \textbf{End-to-End Latency} (\ms): HTTP GET request round-trip time
    % 【中文翻译】端到端延迟(毫秒):HTTP GET请求往返时间
    \item \textbf{Success Rate} (\%): Percentage of successful circuit establishments (target: $\geq$ 99\%)
    % 【中文翻译】成功率(%):成功建立电路的百分比(目标:≥99%)
    \item \textbf{Bandwidth Overhead} (bytes): Onionskin and reply message sizes
    % 【中文翻译】带宽开销(字节):洋葱皮和响应消息大小
\end{itemize}

\paragraph{SAGIN-Specific Metrics}
% 【中文翻译】SAGIN特定指标

\begin{itemize}[leftmargin=*,noitemsep]
    \item \textbf{Handshake Overhead Ratio}: $\text{Hybrid-NTOR\_latency} / \text{Network\_RTT}$ (target: $<$ 1\%)
    % 【中文翻译】握手开销比:Hybrid-NTOR延迟 / 网络RTT(目标:< 1%)
    \item \textbf{Satellite Visibility Window}: Duration satellite is above 10° elevation (calculated using Skyfield with real TLE data)
    % 【中文翻译】卫星可见窗口:卫星高于10°仰角的持续时间(使用Skyfield和真实TLE数据计算)
\end{itemize}

\subsubsection{Experimental Methodology}
% 【中文翻译】实验方法

\paragraph{Phase 1: Isolated Micro-Benchmarks}
% 【中文翻译】第一阶段:隔离微基准测试

\textbf{Objective:} Validate Hybrid-NTOR implementation performance on x86\_64 platform.
% 【中文翻译】目标:验证Hybrid-NTOR在x86_64平台上的实现性能。

\textbf{Setup:} Single-machine testing (no network overhead)
% 【中文翻译】设置:单机测试(无网络开销)

\textbf{Procedure:}
% 【中文翻译】步骤:
\begin{enumerate}[leftmargin=*,noitemsep]
    \item Initialize liboqs library and Hybrid-NTOR state
    % 【中文翻译】初始化liboqs库和Hybrid-NTOR状态
    \item Run 10 warm-up iterations to stabilize CPU cache
    % 【中文翻译】运行10次预热迭代以稳定CPU缓存
    \item Execute 1000 measurement iterations
    % 【中文翻译】执行1000次测量迭代
    \item Compute statistics: min, median, mean, max, standard deviation
    % 【中文翻译】计算统计数据:最小值、中位数、平均值、最大值、标准差
    \item Export results to CSV for analysis
    % 【中文翻译】导出结果到CSV进行分析
\end{enumerate}

\textbf{Validation:} Compare against Berger et al.~\cite{berger2025postquantum} theoretical estimates.
% 【中文翻译】验证:与Berger等人的理论估计进行比较。

\paragraph{Phase 2: SAGIN Network Integration}
% 【中文翻译】第二阶段:SAGIN网络集成

\textbf{Objective:} Test Hybrid-NTOR in simulated space-air-ground networks.
% 【中文翻译】目标:在模拟的空天地网络中测试Hybrid-NTOR。

\textbf{Setup:} 12 network topologies with \texttt{tc/netem} delay simulation
% 【中文翻译】设置:12种网络拓扑,使用tc/netem延迟模拟

\textbf{Procedure} (per topology):
% 【中文翻译】步骤(每个拓扑):
\begin{enumerate}[leftmargin=*,noitemsep]
    \item Deploy network topology using automated scripts
    % 【中文翻译】使用自动化脚本部署网络拓扑
    \item Configure link delays, bandwidth limits, packet loss
    % 【中文翻译】配置链路延迟、带宽限制、丢包率
    \item Start directory server + relay nodes + client
    % 【中文翻译】启动目录服务器 + 中继节点 + 客户端
    \item Wait 5 seconds for network convergence
    % 【中文翻译】等待5秒让网络收敛
    \item Client builds 3-hop circuit using Hybrid-NTOR
    % 【中文翻译】客户端使用Hybrid-NTOR构建三跳电路
    \item Send HTTP GET request, measure CBT and RTT
    % 【中文翻译】发送HTTP GET请求,测量CBT和RTT
    \item Repeat 20 times per topology
    % 【中文翻译】每个拓扑重复20次
    \item Clean up network interfaces
    % 【中文翻译】清理网络接口
\end{enumerate}

\textbf{Total Tests:} $12 \times 20 = 240$ tests
% 【中文翻译】总测试次数:12 × 20 = 240次测试

\textbf{Output:} CSV file with schema: \{timestamp, topo\_id, trial, cbt\_ms, rtt\_ms, success\}
% 【中文翻译】输出:CSV文件,格式:{时间戳, 拓扑ID, 试验次数, CBT毫秒, RTT毫秒, 成功}

\paragraph{Phase 3: Multi-Platform Deployment on Phytium Pi}
% 【中文翻译】第三阶段:在飞腾派上的多平台部署

\textbf{[PLACEHOLDER - Currently in Deployment]}
% 【中文翻译】[占位符 - 正在部署中]

This section will be completed after the Phytium Pi (ARM64) deployment is finalized.
% 【中文翻译】本节将在飞腾派(ARM64)部署完成后补充。

\textbf{Planned Experiments:}
% 【中文翻译】计划的实验:
\begin{itemize}[leftmargin=*,noitemsep]
    \item Distributed 6+1 node deployment (6 relays + 1 control panel)
    % 【中文翻译】分布式6+1节点部署(6个中继 + 1个控制面板)
    \item All 12 topologies executed on ARM64 hardware
    % 【中文翻译】在ARM64硬件上执行所有12种拓扑
    \item Classic NTOR vs Hybrid-NTOR comparison under identical conditions
    % 【中文翻译】在相同条件下比较经典NTOR和Hybrid-NTOR
    \item Performance comparison: x86\_64 (WSL2) vs ARM64 (Phytium Pi)
    % 【中文翻译】性能对比:x86_64(WSL2)vs ARM64(飞腾派)
\end{itemize}

\textbf{Expected Contributions:}
% 【中文翻译】预期贡献:
\begin{itemize}[leftmargin=*,noitemsep]
    \item Validation of Hybrid-NTOR on resource-constrained embedded platforms
    % 【中文翻译】在资源受限的嵌入式平台上验证Hybrid-NTOR
    \item Real-world deployment feasibility assessment
    % 【中文翻译】实际部署可行性评估
    \item ARM64-specific optimizations and bottlenecks identification
    % 【中文翻译】ARM64特定的优化和瓶颈识别
\end{itemize}

\subsection{Phase 1: Cryptographic Performance Benchmarks}
\label{sec:eval:phase1}
% 【中文翻译】5.2 第一阶段:密码学性能基准测试

In Phase 1, we conduct comprehensive micro-benchmarks comparing three handshake protocols: Classic NTOR (X25519 ECDH), PQ-Only NTOR (Kyber-512 KEM only), and Hybrid-NTOR (Kyber-512 + X25519). We evaluate performance on two distinct platforms to assess real-world deployment feasibility.
% 【中文翻译】在第一阶段,我们对三种握手协议进行全面的微基准测试:经典NTOR(X25519 ECDH)、纯后量子NTOR(仅Kyber-512 KEM)和Hybrid-NTOR(Kyber-512 + X25519)。我们在两个不同平台上评估性能,以评估实际部署的可行性。

\subsubsection{Methodology}
% 【中文翻译】测试方法

We implement a rigorous benchmark suite measuring each protocol's performance across multiple operations:
% 【中文翻译】我们实现了一个严格的基准测试套件,测量每个协议在多个操作上的性能:

\begin{itemize}[leftmargin=*,noitemsep]
    \item \textbf{KeyGen:} Key pair generation time
    % 【中文翻译】密钥生成:密钥对生成时间
    \item \textbf{Client Create:} Generate onionskin message
    % 【中文翻译】客户端创建:生成洋葱皮消息
    \item \textbf{Server Reply:} Process onionskin and generate reply
    % 【中文翻译】服务端响应:处理洋葱皮并生成响应
    \item \textbf{Client Finish:} Complete handshake and derive session key
    % 【中文翻译】客户端完成:完成握手并派生会话密钥
    \item \textbf{Full Handshake:} End-to-end single-hop handshake
    % 【中文翻译】完整握手:端到端单跳握手
    \item \textbf{3-Hop Circuit:} Complete Tor circuit establishment (3 sequential handshakes)
    % 【中文翻译】三跳电路:完整的Tor电路建立(3次连续握手)
\end{itemize}

\textbf{Test Parameters:}
% 【中文翻译】测试参数:
\begin{itemize}[leftmargin=*,noitemsep]
    \item Warm-up iterations: 100 (to stabilize CPU cache and branch predictor)
    % 【中文翻译】预热迭代:100次(用于稳定CPU缓存和分支预测器)
    \item Measurement iterations: 1000
    % 【中文翻译】测量迭代:1000次
    \item Time precision: Microsecond (\us) using \texttt{gettimeofday()}
    % 【中文翻译】时间精度:微秒级,使用gettimeofday()
    \item Statistics: Mean, Median, Min, Max, StdDev, P95, P99, 95\% CI
    % 【中文翻译】统计指标:均值、中位数、最小值、最大值、标准差、P95、P99、95%置信区间
    \item Compiler: GCC with \texttt{-O2} optimization
    % 【中文翻译】编译器:GCC,使用-O2优化
\end{itemize}

\subsubsection{Platform Configuration}
% 【中文翻译】平台配置

We conduct experiments on two platforms representing different deployment scenarios:
% 【中文翻译】我们在两个代表不同部署场景的平台上进行实验:

\begin{table}[t]
\centering
\caption{Benchmark Platform Configuration}
\label{tab:bench-platforms}
\small
\begin{tabular}{@{}lll@{}}
\toprule
\textbf{Attribute} & \textbf{x86\_64} & \textbf{ARM64} \\
\midrule
CPU & Intel i9-14900 & Phytium FTC664 \\
Architecture & x86\_64 & aarch64 \\
Kernel & 6.6.87 (WSL2) & 5.10.209 \\
SIMD & AVX2/AVX-512 & NEON \\
\bottomrule
\end{tabular}
\end{table}

\subsubsection{Performance Results}

Table~\ref{tab:protocol-comparison-x86} and Table~\ref{tab:protocol-comparison-arm} present the comprehensive benchmark results for both platforms.

\begin{table}[t]
\centering
\caption{Handshake Performance on WSL2 (x86\_64)}
\label{tab:protocol-comparison-x86}
\footnotesize
\begin{tabular}{@{}llrrrr@{}}
\toprule
\textbf{Protocol} & \textbf{Operation} & \textbf{Mean} & \textbf{Med.} & \textbf{P99} & \textbf{$\sigma$} \\
\midrule
\multirow{3}{*}{Classic} & KeyGen & 25.9 & 25 & 53 & 5.7 \\
 & Handshake & 74.5 & 72 & 119 & 8.0 \\
 & 3-Hop & 221.8 & 216 & 284 & 28.1 \\
\midrule
\multirow{3}{*}{PQ-Only} & KeyGen & 5.8 & 6 & 8 & 1.4 \\
 & Handshake & 32.1 & 31 & 57 & 6.1 \\
 & 3-Hop & 96.7 & 92 & 157 & 15.2 \\
\midrule
\multirow{3}{*}{Hybrid} & KeyGen & 32.1 & 31 & 58 & 6.8 \\
 & Handshake & 192.4 & 183 & 288 & 25.4 \\
 & 3-Hop & 573.8 & 554 & 876 & 66.8 \\
\bottomrule
\end{tabular}
\end{table}

\begin{table}[t]
\centering
\caption{Handshake Performance on Phytium Pi (ARM64)}
\label{tab:protocol-comparison-arm}
\footnotesize
\begin{tabular}{@{}llrrrr@{}}
\toprule
\textbf{Protocol} & \textbf{Operation} & \textbf{Mean} & \textbf{Med.} & \textbf{P99} & \textbf{$\sigma$} \\
\midrule
\multirow{3}{*}{Classic} & KeyGen & 108.3 & 108 & 115 & 6.4 \\
 & Handshake & 457.7 & 456 & 504 & 16.3 \\
 & 3-Hop & 1371.4 & 1367 & 1452 & 15.2 \\
\midrule
\multirow{3}{*}{PQ-Only} & KeyGen & 46.4 & 46 & 53 & 1.0 \\
 & Handshake & 181.0 & 180 & 189 & 5.2 \\
 & 3-Hop & 543.8 & 541 & 621 & 12.1 \\
\midrule
\multirow{3}{*}{Hybrid} & KeyGen & 156.5 & 156 & 164 & 9.9 \\
 & Handshake & 1099.8 & 1096 & 1180 & 12.1 \\
 & 3-Hop & 3300.8 & 3294 & 3385 & 26.9 \\
\bottomrule
\end{tabular}
\end{table}

\subsubsection{Cross-Platform Comparison}
% 【中文翻译】跨平台对比

Table~\ref{tab:cross-platform} summarizes the 3-hop circuit build time (CBT) across both platforms.
% 【中文翻译】表格总结了两个平台上的三跳电路构建时间(CBT)。

\begin{table}[t]
\centering
\caption{3-Hop Circuit Build Time Comparison}
% 【中文翻译】三跳电路构建时间对比
\label{tab:cross-platform}
\small
\begin{tabular}{@{}lrrrr@{}}
\toprule
\textbf{Protocol} & \textbf{WSL2} & \textbf{Phytium} & \textbf{ARM/x86} & \textbf{vs Classic} \\
 & \textbf{(\ms)} & \textbf{(\ms)} & \textbf{Ratio} & \textbf{(ARM)} \\
\midrule
Classic NTOR & 0.222 & 1.371 & 6.18$\times$ & 1.00$\times$ \\
PQ-Only & 0.097 & 0.544 & 5.63$\times$ & \textbf{0.40$\times$} \\
Hybrid NTOR & 0.574 & 3.301 & 5.75$\times$ & 2.41$\times$ \\
\bottomrule
\end{tabular}
\end{table}

\textbf{Key Finding:} PQ-Only NTOR is \textbf{2.5$\times$ faster} than Classic NTOR on both platforms. This counter-intuitive result is due to Kyber-512's efficient lattice-based operations outperforming X25519 elliptic curve scalar multiplication. However, our recommended Hybrid-NTOR provides defense-in-depth with acceptable overhead.
% 【中文翻译】关键发现:纯后量子NTOR在两个平台上都比经典NTOR快2.5倍。这个反直觉的结果是因为Kyber-512高效的基于格的运算优于X25519椭圆曲线标量乘法。然而,我们推荐的Hybrid-NTOR以可接受的开销提供纵深防御。

\subsubsection{Message Size Analysis}
% 【中文翻译】消息大小分析

Table~\ref{tab:message-sizes} presents the communication overhead for each protocol.
% 【中文翻译】表格展示了每个协议的通信开销。

\begin{table}[t]
\centering
\caption{Message Sizes per Hop (bytes)}
% 【中文翻译】每跳消息大小(字节)
\label{tab:message-sizes}
\small
\begin{tabular}{@{}lrrrr@{}}
\toprule
\textbf{Protocol} & \textbf{Onionskin} & \textbf{Reply} & \textbf{Total} & \textbf{vs Classic} \\
\midrule
Classic NTOR & 52 & 64 & 116 & 1.00$\times$ \\
PQ-Only & 820 & 800 & 1,620 & 14.0$\times$ \\
Hybrid NTOR & 852 & 832 & 1,684 & 14.5$\times$ \\
\bottomrule
\end{tabular}
\end{table}

The increased message size (14$\times$) is the primary trade-off for post-quantum security. However, in SAGIN scenarios with typical bandwidths of 10-50 Mbps, this overhead remains negligible:
% 【中文翻译】消息大小增加(14倍)是后量子安全的主要代价。然而,在典型带宽为10-50 Mbps的SAGIN场景中,这种开销仍然可以忽略:
\begin{itemize}[leftmargin=*,noitemsep]
    \item 3-hop circuit total: $1,684 \times 3 = 5,052$ bytes (Hybrid-NTOR)
    % 【中文翻译】三跳电路总计:1,684 × 3 = 5,052字节(Hybrid-NTOR)
    \item At 10 Mbps: transmission time $= 4,860 \times 8 / 10^7 = 3.9$ \ms
    % 【中文翻译】在10 Mbps下:传输时间 = 4,860 × 8 / 10^7 = 3.9毫秒
    \item Compared to typical satellite RTT (50-500 \ms): $<$ 1\% overhead
    % 【中文翻译】与典型卫星RTT(50-500毫秒)相比:< 1%开销
\end{itemize}

\subsubsection{Analysis and Discussion}
% 【中文翻译】分析与讨论

\paragraph{Why is PQ-Only NTOR Faster than Classic NTOR?}
% 【中文翻译】为什么纯后量子NTOR比经典NTOR更快?

The surprising performance advantage of PQ-Only NTOR stems from:
% 【中文翻译】纯后量子NTOR令人惊讶的性能优势源于:

\begin{enumerate}[leftmargin=*,noitemsep]
    \item \textbf{Algorithmic Efficiency:} Kyber-512 uses polynomial ring operations (NTT) which are highly parallelizable, while X25519 requires sequential scalar multiplication on elliptic curves.
    % 【中文翻译】算法效率:Kyber-512使用多项式环运算(NTT),高度可并行化,而X25519需要在椭圆曲线上进行顺序标量乘法。

    \item \textbf{Hardware Optimization:} liboqs implements optimized AVX2/NEON assembly for Kyber, achieving near-theoretical performance bounds.
    % 【中文翻译】硬件优化:liboqs为Kyber实现了优化的AVX2/NEON汇编,达到接近理论的性能上限。

    \item \textbf{Key Size Effect:} Despite larger keys, Kyber's encapsulation/decapsulation operations have lower computational complexity ($O(n \log n)$) compared to ECDH ($O(n^2)$ for naive implementation).
    % 【中文翻译】密钥大小影响:尽管密钥更大,但Kyber的封装/解封装操作具有更低的计算复杂度(O(n log n)),相比ECDH(朴素实现为O(n²))。
\end{enumerate}

\paragraph{Hybrid-NTOR Overhead Analysis}
% 【中文翻译】Hybrid-NTOR开销分析

Hybrid-NTOR exhibits 2.4$\times$ overhead compared to Classic NTOR because it executes \textbf{both} Kyber-512 and X25519 operations sequentially. This provides defense-in-depth: security holds if \textit{either} algorithm remains unbroken.
% 【中文翻译】Hybrid-NTOR相比经典NTOR表现出2.4倍的开销,因为它顺序执行Kyber-512和X25519两种操作。这提供了纵深防御:只要任一算法未被攻破,安全性就能保持。

\paragraph{Platform Performance Gap}
% 【中文翻译】平台性能差距

The ARM64 platform shows 5.6-6.2$\times$ slower performance compared to x86\_64. This is attributed to:
% 【中文翻译】ARM64平台相比x86_64表现出5.6-6.2倍的性能下降。这归因于:
\begin{itemize}[leftmargin=*,noitemsep]
    \item Lower CPU clock frequency on Phytium FTC664
    % 【中文翻译】飞腾FTC664的CPU主频较低
    \item Less aggressive out-of-order execution
    % 【中文翻译】乱序执行能力较弱
    \item NEON SIMD (128-bit) vs AVX2 (256-bit) vector width
    % 【中文翻译】NEON SIMD(128位)vs AVX2(256位)向量宽度
\end{itemize}

However, even on ARM64, all protocols complete handshakes in sub-millisecond time, validating feasibility for edge deployment.
% 【中文翻译】然而,即使在ARM64上,所有协议都能在亚毫秒时间内完成握手,验证了边缘部署的可行性。

\paragraph{Implications for SAGIN Deployment}
% 【中文翻译】对SAGIN部署的启示

Our Phase 1 results demonstrate:
% 【中文翻译】我们的第一阶段结果表明:

\begin{enumerate}[leftmargin=*,noitemsep]
    \item \textbf{Hybrid-NTOR is production-ready:} 3.3 \ms\ 3-hop CBT on ARM64 is negligible compared to satellite RTT (50-500 \ms).
    % 【中文翻译】Hybrid-NTOR已可用于生产:ARM64上3.3毫秒的三跳CBT与卫星RTT(50-500毫秒)相比可忽略不计。

    \item \textbf{Hybrid provides optimal security:} For mission-critical SAGIN applications, the 2.4$\times$ computational overhead (3.3 \ms\ on ARM64) is acceptable.
    % 【中文翻译】混合方案提供最佳安全性:对于关键任务的SAGIN应用,2.4倍的计算开销(ARM64上3.3毫秒)是可接受的。

    \item \textbf{Message size is the primary concern:} Network bandwidth, not computation, is the limiting factor in high-latency environments.
    % 【中文翻译】消息大小是主要关注点:在高延迟环境中,网络带宽而非计算是限制因素。
\end{enumerate}

\subsection{Phase 2: SAGIN Network Integration}
\label{sec:eval:phase2}
% 【中文翻译】5.3 第二阶段:SAGIN网络集成

\textbf{[PLACEHOLDER - To be written after Phase 3 deployment]}
% 【中文翻译】[占位符 - 待第三阶段部署后编写]

This section will present results from 12-topology SAGIN experiments, including:
% 【中文翻译】本节将展示12种拓扑SAGIN实验的结果,包括:
\begin{itemize}[leftmargin=*,noitemsep]
    \item Circuit build time across all topologies
    % 【中文翻译】所有拓扑的电路构建时间
    \item Satellite link delay impact analysis
    % 【中文翻译】卫星链路延迟影响分析
    \item Visibility window calculations with Skyfield
    % 【中文翻译】使用Skyfield计算可见窗口
    \item Comparison with terrestrial baseline
    % 【中文翻译】与地面基线的对比
\end{itemize}

\subsection{Phase 3: Multi-Platform Deployment on Phytium Pi}
\label{sec:eval:phase3}
% 【中文翻译】5.4 第三阶段:在飞腾派上的多平台部署

\textbf{[PLACEHOLDER - Currently in Deployment on Phytium Pi ARM64 Platform]}
% 【中文翻译】[占位符 - 正在飞腾派ARM64平台上部署]

This section will be populated with:
% 【中文翻译】本节将补充以下内容:
\begin{itemize}[leftmargin=*,noitemsep]
    \item Distributed deployment architecture (6+1 nodes)
    % 【中文翻译】分布式部署架构(6+1节点)
    \item Classic NTOR vs Hybrid-NTOR comparison (240 tests)
    % 【中文翻译】经典NTOR与Hybrid-NTOR对比(240次测试)
    \item ARM64 performance analysis
    % 【中文翻译】ARM64性能分析
    \item Real-world deployment lessons learned
    % 【中文翻译】实际部署的经验教训
\end{itemize}

\subsection{Discussion}
\label{sec:eval:discussion}
% 【中文翻译】5.5 讨论

\textbf{[To be written after all phases complete]}
% 【中文翻译】[待所有阶段完成后编写]

Will cover:
% 【中文翻译】将涵盖:
\begin{itemize}[leftmargin=*,noitemsep]
    \item Performance vs. security trade-offs
    % 【中文翻译】性能与安全性的权衡
    \item Real-world deployment feasibility
    % 【中文翻译】实际部署的可行性
    \item Limitations and future work
    % 【中文翻译】局限性和未来工作
    \item Recommendations for Tor network integration
    % 【中文翻译】Tor网络集成建议
\end{itemize}
