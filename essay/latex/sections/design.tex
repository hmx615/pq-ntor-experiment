% Section 4: Hybrid-NTOR Design
% 【中文翻译】第四章:Hybrid-NTOR设计

\section{Hybrid-NTOR Protocol Design}
\label{sec:design}
% 【中文翻译】第四章 Hybrid-NTOR协议设计

\subsection{Protocol Specification}
\label{sec:protocol-spec}
% 【中文翻译】4.1 协议规范

The Hybrid-NTOR handshake protocol extends the original Tor NTOR protocol by combining Kyber-512 KEM with X25519 ECDH, following IETF recommendations for hybrid post-quantum key exchange~\cite{ietf-hybrid-kem}. This hybrid approach provides defense-in-depth: Kyber-512 ensures quantum resistance while X25519 maintains classical security guarantees. Even if one cryptographic primitive is compromised, the other continues to protect the session key.
% 【中文翻译】Hybrid-NTOR握手协议通过结合Kyber-512 KEM与X25519 ECDH扩展了原始的Tor NTOR协议,遵循IETF关于混合后量子密钥交换的建议。这种混合方法提供纵深防御:Kyber-512确保抗量子性,而X25519保持经典安全保证。即使一个密码学原语被攻破,另一个仍能继续保护会话密钥。

\subsubsection{Design Rationale}
% 【中文翻译】设计原理

The hybrid design is motivated by three considerations:
% 【中文翻译】混合设计基于三方面考量:

\begin{enumerate}[leftmargin=*, label=(\arabic*)]
    \item It addresses the ``harvest now, decrypt later'' threat model, where adversaries collect encrypted Tor traffic today for decryption once cryptographically relevant quantum computers become available. By incorporating Kyber-512, the protocol ensures that recorded sessions remain confidential even against future quantum adversaries.
    % 【中文翻译】应对"先存储后解密"威胁模型,即攻击者收集当前加密的Tor流量,待密码学相关的量子计算机可用后再解密。通过引入Kyber-512,协议确保已记录的会话即使面对未来量子攻击者也能保持机密性。

    \item The hybrid approach provides cryptographic agility during the ongoing post-quantum transition period. While Kyber has undergone extensive cryptanalysis during NIST standardization, lattice-based schemes have a shorter deployment history than elliptic curve cryptography. Retaining X25519 ensures that session security remains intact as long as at least one primitive remains unbroken.
    % 【中文翻译】混合方案在当前后量子过渡期提供密码学敏捷性。虽然Kyber在NIST标准化过程中经历了广泛的密码分析,但基于格的方案部署历史比椭圆曲线密码学更短。保留X25519确保只要至少一个原语未被攻破,会话安全性就能保持完整。

    \item The design maintains compatibility with Tor's existing circuit construction semantics. The handshake message flow remains a two-message exchange, preserving the same round-trip structure as classic NTOR and requiring no changes to Tor's cell relay protocol.
    % 【中文翻译】该设计保持与Tor现有电路构建语义的兼容性。握手消息流仍为两消息交换,保留与经典NTOR相同的往返结构,无需修改Tor的cell中继协议。
\end{enumerate}

\subsubsection{Notation}
% 【中文翻译】符号说明
\begin{itemize}[leftmargin=*]
    \item $(pk_K, sk_K)$: Client's ephemeral Kyber-512 key pair
    % 【中文翻译】客户端的临时Kyber-512密钥对
    \item $(X, x)$: Client's ephemeral X25519 key pair
    % 【中文翻译】客户端的临时X25519密钥对
    \item $(Y, y)$: Relay's ephemeral X25519 key pair
    % 【中文翻译】中继的临时X25519密钥对
    \item $\text{Encap}(pk_K) \rightarrow (ct, ss_K)$: Kyber-512 encapsulation
    % 【中文翻译】Kyber-512封装操作
    \item $\text{Decap}(sk_K, ct) \rightarrow ss_K$: Kyber-512 decapsulation
    % 【中文翻译】Kyber-512解封装操作
    \item $ss_K$, $ss_X$, $ss_{hybrid}$: Shared secrets from Kyber, X25519, and their combination
    % 【中文翻译】来自Kyber、X25519及其组合的共享密钥
    \item $ID_R$: Relay identity
    % 【中文翻译】中继身份标识
    \item $KDF(\cdot)$: HKDF-SHA256 key derivation function
    % 【中文翻译】HKDF-SHA256密钥派生函数
    \item $HMAC(\cdot)$: HMAC-SHA256 message authentication code
    % 【中文翻译】HMAC-SHA256消息认证码
    \item $E_{K_i}(\cdot)$: Encryption with session key $K_i$
    % 【中文翻译】用会话密钥$K_i$加密
    \item $o_i$, $r_i$: Onionskin and reply message for relay $i$
    % 【中文翻译】发给中继$i$的onionskin和回复消息
\end{itemize}

\noindent Key sizes: Kyber-512 public key 800 bytes, secret key 1632 bytes, ciphertext 768 bytes; X25519 keys 32 bytes each; all shared secrets and $ID_R$ are 32 and 20 bytes respectively.
% 【中文翻译】密钥大小:Kyber-512公钥800字节,私钥1632字节,密文768字节;X25519密钥各32字节;所有共享密钥和$ID_R$分别为32和20字节。

\subsubsection{Protocol Flow}
% 【中文翻译】协议流程

Figure~\ref{fig:hybrid-ntor-handshake} illustrates the Hybrid-NTOR handshake protocol. The protocol consists of two messages exchanged between client and relay, establishing a hybrid shared secret from both Kyber KEM and X25519 ECDH.
% 【中文翻译】图~\ref{fig:hybrid-ntor-handshake}展示了Hybrid-NTOR握手协议。该协议由客户端和中继之间交换的两条消息组成,从Kyber KEM和X25519 ECDH建立混合共享密钥。

\begin{figure}[t]
\centering
\small
\begin{tikzpicture}[
    node distance=0.4cm,
    actor/.style={rectangle, draw, minimum width=1.8cm, minimum height=0.5cm, align=center, font=\small\bfseries},
    opbox/.style={rectangle, draw=black, minimum width=2.5cm, align=left, font=\scriptsize, inner sep=3pt},
    arrow/.style={->, >=stealth, thick},
    msg/.style={font=\scriptsize}
]

% Actors
\node[actor] (client) {Client};
\node[actor, right=4.5cm of client] (relay) {Relay};

% Vertical timelines
\coordinate (c0) at ($(client.south) + (0,-0.25)$);
\coordinate (r0) at ($(relay.south) + (0,-0.25)$);

% Step 1: Client generates keypairs
\node[opbox, below=0.35cm of c0, anchor=north] (box1) {
    $(pk_K, sk_K) \leftarrow \text{Kyber.KeyGen}()$ \\
    $(X, x) \leftarrow \text{X25519.KeyGen}()$
};
\draw[thick] (c0) -- (box1.north);

% Message 1: Client -> Relay
\coordinate (c1) at ($(box1.south) + (0,-0.25)$);
\coordinate (r1) at ($(r0) + (0,-2.0)$);
\draw[thick] (box1.south) -- (c1);
\draw[arrow] (c1) -- (r1) node[midway, above, msg] {$\{pk_K, X, ID_R\}$};
\draw[thick] (r0) -- (r1);

% Step 2: Relay processes and responds
\node[opbox, below=0.35cm of r1, anchor=north] (box2) {
    $(ct, ss_K) \leftarrow \text{Encap}(pk_K)$ \\
    $(Y, y) \leftarrow \text{X25519.KeyGen}()$ \\
    $ss_X \leftarrow y \cdot X$ \\
    $ss_{hyb} \leftarrow KDF(ss_K \| ss_X)$ \\
    $AUTH \leftarrow HMAC(K_{auth}, T)$
};
\draw[thick] (r1) -- (box2.north);

% Message 2: Relay -> Client
\coordinate (r2) at ($(box2.south) + (0,-0.25)$);
\coordinate (c2) at ($(c1) + (0,-5.0)$);
\draw[thick] (box2.south) -- (r2);
\draw[arrow] (r2) -- (c2) node[pos=0.3, above, msg] {$\{ct, Y, AUTH\}$};
\draw[thick] (c1) -- (c2);

% Step 3: Client verifies
\node[opbox, below=0.35cm of c2, anchor=north] (box3) {
    $ss_K \leftarrow \text{Decap}(sk_K, ct)$ \\
    $ss_X \leftarrow x \cdot Y$ \\
    $ss_{hyb} \leftarrow KDF(ss_K \| ss_X)$ \\
    Verify: $AUTH' \stackrel{?}{=} AUTH$
};
\draw[thick] (c2) -- (box3.north);

% Final timeline endpoints
\coordinate (c_end) at ($(box3.south) + (0,-0.15)$);
\coordinate (r_end) at (r2 |- c_end);
\draw[thick] (box3.south) -- (c_end);
\draw[thick] (r2) -- (r_end);

\end{tikzpicture}
\caption{Hybrid-NTOR handshake protocol combining Kyber-512 KEM with X25519 ECDH.}
% 【中文翻译】结合Kyber-512 KEM与X25519 ECDH的Hybrid-NTOR握手协议。
\label{fig:hybrid-ntor-handshake}
\end{figure}

\paragraph{Step 1: Client Initiates Handshake}
% 【中文翻译】步骤1:客户端发起握手

The client generates two independent ephemeral key pairs:
% 【中文翻译】客户端生成两个独立的临时密钥对:
\begin{align}
(pk_K, sk_K) &\leftarrow \text{Kyber.KeyGen}() \\
(X, x) &\leftarrow \text{X25519.KeyGen}()
\end{align}

The client constructs and sends the \emph{onionskin} message (852 bytes):
% 【中文翻译】客户端构建并发送onionskin消息(852字节):
\begin{equation}
C \rightarrow R: \{pk_K \| X \| ID_R\}
\end{equation}

where $pk_K$ is the Kyber public key (800 bytes), $X$ is the X25519 public key (32 bytes), and $ID_R$ is the relay identity (20 bytes).
% 【中文翻译】其中$pk_K$是Kyber公钥(800字节),$X$是X25519公钥(32字节),$ID_R$是中继身份标识(20字节)。

\paragraph{Step 2: Relay Processes and Responds}
% 【中文翻译】步骤2:中继处理并响应

Upon receiving the onionskin, the relay performs the following operations:
% 【中文翻译】收到onionskin后,中继执行以下操作:

\begin{enumerate}[leftmargin=*]
    \item \textbf{Kyber Encapsulation}: Generate ciphertext and Kyber shared secret:
    % 【中文翻译】Kyber封装:生成密文和Kyber共享密钥:
    \begin{equation}
    (ct, ss_K) \leftarrow \text{Encap}(pk_K)
    \end{equation}

    \item \textbf{X25519 Key Exchange}: Generate ephemeral key pair and compute ECDH shared secret:
    % 【中文翻译】X25519密钥交换:生成临时密钥对并计算ECDH共享密钥:
    \begin{align}
    (Y, y) &\leftarrow \text{X25519.KeyGen}() \\
    ss_X &\leftarrow y \cdot X
    \end{align}

    \item \textbf{Hybrid Secret Combination}: Derive hybrid shared secret using HKDF:
    % 【中文翻译】混合密钥组合:使用HKDF派生混合共享密钥:
    \begin{equation}
    ss_{hybrid} \leftarrow KDF(ss_K \| ss_X, \text{``hybrid-ntor-combine''})
    \end{equation}

    \item \textbf{Session Key Derivation}: Derive authentication and encryption keys:
    % 【中文翻译】会话密钥派生:派生认证密钥和加密密钥:
    \begin{equation}
    (K_{auth}, K_{enc}) \leftarrow KDF(ss_{hybrid}, H(T), \text{``hybrid-ntor-keys''})
    \end{equation}
    where transcript $T = pk_K \| X \| ct \| Y \| ID_R$.
    % 【中文翻译】其中会话记录$T = pk_K \| X \| ct \| Y \| ID_R$。

    \item \textbf{Authentication}: Compute authentication tag:
    % 【中文翻译】认证:计算认证标签:
    \begin{equation}
    AUTH \leftarrow HMAC(K_{auth}, T \| \text{``server''})
    \end{equation}
\end{enumerate}

The relay sends the \emph{reply} message (832 bytes):
% 【中文翻译】中继发送reply消息(832字节):
\begin{equation}
R \rightarrow C: \{ct \| Y \| AUTH\}
\end{equation}

where $ct$ is the Kyber ciphertext (768 bytes), $Y$ is the X25519 public key (32 bytes), and $AUTH$ is the authentication tag (32 bytes).
% 【中文翻译】其中$ct$是Kyber密文(768字节),$Y$是X25519公钥(32字节),$AUTH$是认证标签(32字节)。

\paragraph{Step 3: Client Completes Handshake}
% 【中文翻译】步骤3:客户端完成握手

The client processes the reply to complete the handshake:
% 【中文翻译】客户端处理reply以完成握手:

\begin{enumerate}[leftmargin=*]
    \item \textbf{Kyber Decapsulation}: Recover Kyber shared secret:
    % 【中文翻译】Kyber解封装:恢复Kyber共享密钥:
    \begin{equation}
    ss_K \leftarrow \text{Decap}(sk_K, ct)
    \end{equation}

    \item \textbf{X25519 Key Exchange}: Compute ECDH shared secret:
    % 【中文翻译】X25519密钥交换:计算ECDH共享密钥:
    \begin{equation}
    ss_X \leftarrow x \cdot Y
    \end{equation}

    \item \textbf{Hybrid Secret and Key Derivation}: Same as relay:
    % 【中文翻译】混合密钥和会话密钥派生:与中继相同:
    \begin{align}
    ss_{hybrid} &\leftarrow KDF(ss_K \| ss_X, \text{``hybrid-ntor-combine''}) \\
    (K_{auth}, K_{enc}) &\leftarrow KDF(ss_{hybrid}, H(T), \text{``hybrid-ntor-keys''})
    \end{align}

    \item \textbf{Verification}: Compute and verify authentication tag:
    % 【中文翻译】验证:计算并验证认证标签:
    \begin{equation}
    AUTH' \leftarrow HMAC(K_{auth}, T \| \text{``server''})
    \end{equation}
    Accept if $AUTH' = AUTH$ (constant-time comparison).
    % 【中文翻译】如果$AUTH' = AUTH$则接受(常量时间比较)。
\end{enumerate}

Upon successful verification, both parties share identical $K_{enc}$ (80 bytes) for subsequent circuit encryption, comprising forward key $K_f$ (32 bytes), backward key $K_b$ (32 bytes), and initialization vectors (16 bytes).
% 【中文翻译】验证成功后,双方共享相同的$K_{enc}$(80字节)用于后续电路加密,包括前向密钥$K_f$(32字节)、后向密钥$K_b$(32字节)和初始化向量(16字节)。

\subsubsection{Message Format Summary}
% 【中文翻译】消息格式总结

Table~\ref{tab:message-format} summarizes the message sizes for the Hybrid-NTOR protocol compared with Classic NTOR.
% 【中文翻译】表~\ref{tab:message-format}总结了Hybrid-NTOR协议与经典NTOR的消息大小比较。

\begin{table}[h]
\centering
\small
\begin{tabular}{@{}lccc@{}}
\toprule
\textbf{Message} & \textbf{Classic NTOR} & \textbf{Hybrid-NTOR} & \textbf{Overhead} \\
\midrule
Onionskin & 52 bytes & 852 bytes & +800 bytes \\
Reply & 64 bytes & 832 bytes & +768 bytes \\
\midrule
Total & 116 bytes & 1684 bytes & +1568 bytes \\
\bottomrule
\end{tabular}
\caption{Message size comparison between Classic NTOR and Hybrid-NTOR.}
% 【中文翻译】经典NTOR与Hybrid-NTOR的消息大小比较。
\label{tab:message-format}
\end{table}

The increased message size is primarily due to Kyber-512's larger key (800 bytes) and ciphertext (768 bytes). However, this overhead is acceptable in modern networks and negligible compared to typical SAGIN round-trip times (50-600 ms).
% 【中文翻译】消息大小增加主要由于Kyber-512较大的密钥(800字节)和密文(768字节)。然而,这种开销在现代网络中是可接受的,与典型的SAGIN往返时间(50-600毫秒)相比可忽略不计。

\subsubsection{Three-Hop Circuit Construction}
% 【中文翻译】三跳电路构建

A complete Tor circuit requires establishing shared keys with three relays: Guard ($R_1$), Middle ($R_2$), and Exit ($R_3$). Figure~\ref{fig:circuit-construction} illustrates the circuit building process using Hybrid-NTOR handshakes.
% 【中文翻译】完整的Tor电路需要与三个中继建立共享密钥:守卫中继($R_1$)、中间中继($R_2$)和出口中继($R_3$)。图~\ref{fig:circuit-construction}展示了使用Hybrid-NTOR握手的电路建立过程。

\begin{figure}[t]
\centering
\footnotesize
\begin{tikzpicture}[
    actor/.style={rectangle, draw, minimum width=0.9cm, minimum height=0.4cm, align=center, font=\footnotesize\bfseries},
    arrow/.style={->, >=stealth, semithick},
    msg/.style={font=\tiny, fill=white, inner sep=1pt},
    note/.style={font=\tiny},
    phase1box/.style={draw=blue!70, densely dashed, rounded corners=2pt},
    phase2box/.style={draw=red!70, densely dashed, rounded corners=2pt},
    phase3box/.style={draw=green!60!black, densely dashed, rounded corners=2pt}
]

% Define x-coordinates for each actor's timeline (wider spacing)
\def\xC{0}
\def\xG{1.8}
\def\xM{3.6}
\def\xE{5.4}

% Actors
\node[actor] at (\xC, 0) (client) {Client};
\node[actor] at (\xG, 0) (guard) {$R_1$};
\node[actor] at (\xM, 0) (middle) {$R_2$};
\node[actor] at (\xE, 0) (exit) {$R_3$};

% Labels
\node[above=0.05cm of guard, font=\tiny] {Guard};
\node[above=0.05cm of middle, font=\tiny] {Middle};
\node[above=0.05cm of exit, font=\tiny] {Exit};

% Define y-coordinates for messages (negative = downward, more spacing)
\def\yStart{-0.4}
% Phase 1
\def\yPa{-0.7}   % P1 msg1
\def\yPb{-1.2}   % P1 msg2
\def\yPend{-1.5} % P1 end
% Phase 2 (more gap from Phase 1)
\def\yPTa{-2.2}  % P2 msg1
\def\yPTb{-2.7}  % P2 msg2
\def\yPTc{-3.2}  % P2 msg3
\def\yPTd{-3.7}  % P2 msg4
\def\yPTend{-4.0} % P2 end
% Phase 3 (more gap from Phase 2)
\def\yPHa{-4.7}  % P3 msg1
\def\yPHb{-5.2}  % P3 msg2
\def\yPHc{-5.7}  % P3 msg3
\def\yPHd{-6.2}  % P3 msg4
\def\yPHe{-6.7}  % P3 msg5
\def\yPHf{-7.2}  % P3 msg6
\def\yPHend{-7.5} % P3 end

% Draw vertical timelines FIRST (so arrows are on top)
\draw[thick] (\xC, \yStart) -- (\xC, \yPHend);
\draw[thick] (\xG, \yStart) -- (\xG, \yPHend);
\draw[thick] (\xM, \yStart) -- (\xM, \yPHend);
\draw[thick] (\xE, \yStart) -- (\xE, \yPHend);

% ===== Phase 1: Client <-> Guard =====
\draw[arrow] (\xC, \yPa) -- (\xG, \yPa) node[midway, above, msg] {$\{pk_{K1}, X_1\}$};
\draw[arrow] (\xG, \yPb) -- (\xC, \yPb) node[midway, above, msg] {$\{ct_1, Y_1\}$};
\node[note] at (\xG+0.6, \yPb) {$K_1$};

% Phase 1 box
\draw[phase1box] (\xC-0.6, \yStart+0.05) rectangle (\xG+0.45, \yPend);
\node[font=\tiny\bfseries, blue!70] at (\xC-0.8, {(\yStart+\yPend)/2}) [rotate=90] {Phase 1};

% ===== Phase 2: Client <-> Middle via Guard =====
\draw[arrow] (\xC, \yPTa) -- (\xG, \yPTa) node[midway, above, msg] {$E_{K_1}(o_2)$};
\draw[arrow] (\xG, \yPTb) -- (\xM, \yPTb) node[midway, above, msg] {$\{pk_{K2}, X_2\}$};
\draw[arrow] (\xM, \yPTc) -- (\xG, \yPTc) node[midway, above, msg] {$\{ct_2, Y_2\}$};
\draw[arrow] (\xG, \yPTd) -- (\xC, \yPTd) node[midway, above, msg] {$E_{K_1}(r_2)$};
\node[note] at (\xM+0.6, \yPTc) {$K_2$};

% Phase 2 box
\draw[phase2box] (\xC-0.6, \yPTa+0.3) rectangle (\xM+0.45, \yPTend);
\node[font=\tiny\bfseries, red!70] at (\xC-0.8, {(\yPTa+0.3+\yPTend)/2}) [rotate=90] {Phase 2};

% ===== Phase 3: Client <-> Exit via Guard, Middle =====
\draw[arrow] (\xC, \yPHa) -- (\xG, \yPHa) node[midway, above, msg] {$E_{K_1}(E_{K_2}(o_3))$};
\draw[arrow] (\xG, \yPHb) -- (\xM, \yPHb) node[midway, above, msg] {$E_{K_2}(o_3)$};
\draw[arrow] (\xM, \yPHc) -- (\xE, \yPHc) node[midway, above, msg] {$\{pk_{K3}, X_3\}$};
\draw[arrow] (\xE, \yPHd) -- (\xM, \yPHd) node[midway, above, msg] {$\{ct_3, Y_3\}$};
\draw[arrow] (\xM, \yPHe) -- (\xG, \yPHe) node[midway, above, msg] {$E_{K_2}(r_3)$};
\draw[arrow] (\xG, \yPHf) -- (\xC, \yPHf) node[midway, above, msg] {$E_{K_1}(E_{K_2}(r_3))$};
\node[note] at (\xE+0.5, \yPHd) {$K_3$};

% Phase 3 box
\draw[phase3box] (\xC-0.6, \yPHa+0.3) rectangle (\xE+0.4, \yPHend);
\node[font=\tiny\bfseries, green!60!black] at (\xC-0.8, {(\yPHa+0.3+\yPHend)/2}) [rotate=90] {Phase 3};

\end{tikzpicture}
\caption{Three-hop circuit construction using Hybrid-NTOR.}
% 【中文翻译】使用Hybrid-NTOR的三跳电路构建。
\label{fig:circuit-construction}
\end{figure}

The circuit construction proceeds in three sequential phases:
% 【中文翻译】电路构建分三个顺序阶段进行:

\begin{enumerate}[leftmargin=*]
    \item \textbf{Phase 1 (Guard)}: Client performs Hybrid-NTOR handshake directly with $R_1$, establishing session key $K_1$. This requires one RTT.
    % 【中文翻译】阶段1(守卫中继):客户端直接与$R_1$执行Hybrid-NTOR握手,建立会话密钥$K_1$。这需要一个RTT。

    \item \textbf{Phase 2 (Middle)}: Client sends Hybrid-NTOR onionskin to $R_2$ through $R_1$, encrypted with $K_1$. $R_1$ decrypts and forwards to $R_2$. The reply is encrypted by $R_2$ and re-encrypted by $R_1$ before reaching client. Session key $K_2$ is established.
    % 【中文翻译】阶段2(中间中继):客户端通过$R_1$向$R_2$发送Hybrid-NTOR onionskin,用$K_1$加密。$R_1$解密后转发给$R_2$。回复由$R_2$加密,$R_1$再次加密后到达客户端。建立会话密钥$K_2$。

    \item \textbf{Phase 3 (Exit)}: Client sends doubly-encrypted onionskin through the established path. Each relay removes one encryption layer. Session key $K_3$ is established with the exit relay.
    % 【中文翻译】阶段3(出口中继):客户端通过已建立的路径发送双重加密的onionskin。每个中继移除一层加密。与出口中继建立会话密钥$K_3$。
\end{enumerate}

The total Circuit Building Time (CBT) is:
% 【中文翻译】总电路建立时间(CBT)为:
\begin{equation}
T_{\text{CBT}} = \sum_{i=1}^{3} (T_{\text{crypto},i} + \text{RTT}_i)
\label{eq:cbt}
\end{equation}

where $T_{\text{crypto},i}$ includes Hybrid-NTOR computation at both client and relay $R_i$, and $\text{RTT}_i$ is the round-trip time to relay $R_i$ through the partial circuit. In SAGIN environments, the accumulated RTT across satellite links dominates the total CBT.
% 【中文翻译】其中$T_{\text{crypto},i}$包括客户端和中继$R_i$的Hybrid-NTOR计算,$\text{RTT}_i$是通过部分电路到中继$R_i$的往返时间。在SAGIN环境中,跨卫星链路的累积RTT主导总CBT。

\subsection{Security Properties}
\label{sec:security-properties}
% 【中文翻译】4.2 安全属性

We provide formal security analysis of Hybrid-NTOR using BAN Logic (Burrows-Abadi-Needham Logic), a widely-adopted framework for analyzing authentication protocols. BAN Logic allows us to reason about beliefs and knowledge of protocol participants, proving that the protocol achieves mutual authentication, key freshness, and secrecy.
% 【中文翻译】我们使用BAN逻辑(Burrows-Abadi-Needham逻辑)对Hybrid-NTOR进行形式化安全分析,这是一个广泛采用的认证协议分析框架。BAN逻辑允许我们推理协议参与者的信念和知识,证明协议实现了双向认证、密钥新鲜性和保密性。

\subsubsection{BAN Logic Preliminaries}
% 【中文翻译】BAN逻辑预备知识

BAN Logic uses the following notation:
% 【中文翻译】BAN逻辑使用以下符号:
\begin{itemize}[leftmargin=*]
    \item $P \believes X$: Principal $P$ believes statement $X$
    % 【中文翻译】主体P相信陈述X
    \item $P \sees X$: Principal $P$ has received message $X$
    % 【中文翻译】主体P已收到消息X
    \item $P \said X$: Principal $P$ once said $X$
    % 【中文翻译】主体P曾经说过X
    \item $P \controls X$: Principal $P$ has jurisdiction over $X$
    % 【中文翻译】主体P对X具有管辖权
    \item $\#(X)$: Message $X$ is fresh (recently generated)
    % 【中文翻译】消息X是新鲜的(最近生成的)
    \item $P \xleftrightarrow{K} Q$: $P$ and $Q$ share secret key $K$
    % 【中文翻译】P和Q共享密钥K
    \item $\{X\}_K$: Message $X$ encrypted with key $K$
    % 【中文翻译】用密钥K加密的消息X
    \item $\langle X \rangle_Y$: $X$ combined with $Y$ (e.g., hashing)
    % 【中文翻译】X与Y结合(例如,哈希)
\end{itemize}

\textbf{BAN Logic Inference Rules:}
% 【中文翻译】BAN逻辑推理规则:
\begin{enumerate}[leftmargin=*]
    \item \textbf{Message-Meaning Rule (Public Key):}
    % 【中文翻译】消息含义规则(公钥):
    \[\frac{P \believes Q \xleftrightarrow{K} P, \quad P \sees \{X\}_K}{P \believes Q \said X}\]

    \item \textbf{Nonce-Verification Rule:}
    % 【中文翻译】随机数验证规则:
    \[\frac{P \believes \#(X), \quad P \believes Q \said X}{P \believes Q \believes X}\]

    \item \textbf{Jurisdiction Rule:}
    % 【中文翻译】管辖权规则:
    \[\frac{P \believes Q \controls X, \quad P \believes Q \believes X}{P \believes X}\]

    \item \textbf{Freshness-Conjunction Rule:}
    % 【中文翻译】新鲜性合取规则:
    \[\frac{P \believes \#(X)}{P \believes \#(X, Y)}\]

    \item \textbf{Belief-Conjunction Rule:}
    % 【中文翻译】信念合取规则:
    \[\frac{P \believes X, \quad P \believes Y}{P \believes (X, Y)}\]
\end{enumerate}

\subsubsection{Idealized Protocol}
% 【中文翻译】理想化协议

We first transform the Hybrid-NTOR protocol into BAN idealized form:
% 【中文翻译】我们首先将Hybrid-NTOR协议转换为BAN理想化形式:

\textbf{Message 1 (Client to Relay):}
% 【中文翻译】消息1(客户端到中继):
\begin{equation}
C \rightarrow R: \{N_C, ss\}_{pk_R}
\end{equation}

\textbf{Message 2 (Relay to Client):}
% 【中文翻译】消息2(中继到客户端):
\begin{equation}
R \rightarrow C: \{N_R, K_{session}\}_{ss}
\end{equation}

Where:
% 【中文翻译】其中:
\begin{itemize}[leftmargin=*]
    \item $\{N_C, ss\}_{pk_R}$ represents Kyber encapsulation (ciphertext $ct$)
    % 【中文翻译】表示Kyber封装(密文ct)
    \item $\{N_R, K_{session}\}_{ss}$ represents the authenticated response
    % 【中文翻译】表示认证的响应
    \item $K_{session} = KDF(ss \| N_C \| N_R \| pk_R)$
    % 【中文翻译】会话密钥派生公式
\end{itemize}

\subsubsection{Initial Assumptions}
% 【中文翻译】初始假设

\textbf{A1.} $C \believes \#(N_C)$ \hfill (Client's nonce is fresh)
% 【中文翻译】客户端的随机数是新鲜的

\textbf{A2.} $R \believes \#(N_R)$ \hfill (Relay's nonce is fresh)
% 【中文翻译】中继的随机数是新鲜的

\textbf{A3.} $C \believes R \xleftrightarrow{pk_R} C$ \hfill (Client trusts relay's public key)
% 【中文翻译】客户端信任中继的公钥

\textbf{A4.} $R \believes R \xleftrightarrow{pk_R} C$ \hfill (Relay owns its key pair)
% 【中文翻译】中继拥有其密钥对

\textbf{A5.} $C \believes R \controls K_{session}$ \hfill (Relay controls session key)
% 【中文翻译】中继控制会话密钥

\textbf{A6.} $R \believes C \controls ss$ \hfill (Client controls shared secret generation)
% 【中文翻译】客户端控制共享密钥生成

\subsubsection{Security Goals}
% 【中文翻译】安全目标

We prove the following security goals:
% 【中文翻译】我们证明以下安全目标:

\textbf{G1.} $C \believes C \xleftrightarrow{K_{session}} R$ \hfill (Client believes in session key)
% 【中文翻译】客户端相信会话密钥

\textbf{G2.} $R \believes C \xleftrightarrow{K_{session}} R$ \hfill (Relay believes in session key)
% 【中文翻译】中继相信会话密钥

\textbf{G3.} $C \believes \#(K_{session})$ \hfill (Client believes key is fresh)
% 【中文翻译】客户端相信密钥是新鲜的

\textbf{G4.} $R \believes \#(K_{session})$ \hfill (Relay believes key is fresh)
% 【中文翻译】中继相信密钥是新鲜的

\subsubsection{Formal Proof}
% 【中文翻译】形式化证明

\textbf{Proof of G2: Relay believes in session key}
% 【中文翻译】证明G2:中继相信会话密钥

\textbf{Step 1.} From Message 1, the relay sees:
% 【中文翻译】步骤1:从消息1,中继看到:
\[R \sees \{N_C, ss\}_{pk_R}\]

\textbf{Step 2.} By decapsulation with $sk_R$ and assumption \textbf{A4}:
% 【中文翻译】步骤2:通过使用sk_R解封装和假设A4:
\[R \believes C \said (N_C, ss)\]
(Message-Meaning Rule)
% 【中文翻译】(消息含义规则)

\textbf{Step 3.} From assumption \textbf{A1} and Freshness-Conjunction:
% 【中文翻译】步骤3:从假设A1和新鲜性合取:
\[R \believes \#(N_C, ss)\]

\textbf{Step 4.} Applying Nonce-Verification Rule to Steps 2 and 3:
% 【中文翻译】步骤4:对步骤2和3应用随机数验证规则:
\[R \believes C \believes (N_C, ss)\]

\textbf{Step 5.} The relay generates fresh $N_R$ (assumption \textbf{A2}):
% 【中文翻译】步骤5:中继生成新鲜的N_R(假设A2):
\[R \believes \#(N_R)\]

\textbf{Step 6.} The relay computes:
% 【中文翻译】步骤6:中继计算:
\[K_{session} = KDF(ss \| N_C \| N_R \| pk_R)\]

Since $N_R$ is fresh and $ss$ is authenticated:
% 【中文翻译】由于N_R是新鲜的且ss已认证:
\[R \believes \#(K_{session})\]
This proves \textbf{G4}.
% 【中文翻译】这证明了G4。

\textbf{Step 7.} From assumption \textbf{A6} and Step 4:
% 【中文翻译】步骤7:从假设A6和步骤4:
\[R \believes C \xleftrightarrow{K_{session}} R\]
This proves \textbf{G2}.
% 【中文翻译】这证明了G2。

\textbf{Proof of G1: Client believes in session key}
% 【中文翻译】证明G1:客户端相信会话密钥

\textbf{Step 8.} From Message 2, the client sees:
% 【中文翻译】步骤8:从消息2,客户端看到:
\[C \sees \{N_R, K_{session}\}_{ss}\]

\textbf{Step 9.} Since the client generated $ss$ via encapsulation and $ss$ is secret:
% 【中文翻译】步骤9:由于客户端通过封装生成了ss且ss是保密的:
\[C \believes R \said (N_R, K_{session})\]
(Message-Meaning Rule with shared secret $ss$)
% 【中文翻译】(使用共享密钥ss的消息含义规则)

\textbf{Step 10.} The relay's response includes fresh $N_R$:
% 【中文翻译】步骤10:中继的响应包含新鲜的N_R:
\[C \believes \#(N_R)\]

\textbf{Step 11.} Applying Freshness-Conjunction:
% 【中文翻译】步骤11:应用新鲜性合取:
\[C \believes \#(N_R, K_{session})\]

Therefore:
% 【中文翻译】因此:
\[C \believes \#(K_{session})\]
This proves \textbf{G3}.
% 【中文翻译】这证明了G3。

\textbf{Step 12.} Applying Nonce-Verification Rule to Steps 9 and 11:
% 【中文翻译】步骤12:对步骤9和11应用随机数验证规则:
\[C \believes R \believes K_{session}\]

\textbf{Step 13.} From assumption \textbf{A5} (Jurisdiction Rule):
% 【中文翻译】步骤13:从假设A5(管辖权规则):
\[C \believes K_{session}\]

\textbf{Step 14.} Since the client computed $K_{session}$ using authenticated $ss$ and fresh nonces:
% 【中文翻译】步骤14:由于客户端使用已认证的ss和新鲜的随机数计算了K_{session}:
\[C \believes C \xleftrightarrow{K_{session}} R\]
This proves \textbf{G1}.
% 【中文翻译】这证明了G1。

\subsubsection{Security Properties Achieved}
% 【中文翻译】实现的安全属性

The BAN Logic proof establishes:
% 【中文翻译】BAN逻辑证明建立了:

\begin{enumerate}[leftmargin=*]
    \item \textbf{Mutual Authentication:} Both client and relay believe they share the session key with the intended party (\textbf{G1}, \textbf{G2})
    % 【中文翻译】双向认证:客户端和中继都相信他们与预期方共享会话密钥(G1,G2)

    \item \textbf{Key Freshness:} Both parties believe the session key is fresh (\textbf{G3}, \textbf{G4}), preventing replay attacks
    % 【中文翻译】密钥新鲜性:双方都相信会话密钥是新鲜的(G3,G4),防止重放攻击

    \item \textbf{Key Agreement:} The protocol achieves secure key establishment where:
    % 【中文翻译】密钥协商:协议实现了安全密钥建立,其中:
    \[C \believes (C \xleftrightarrow{K_{session}} R) \wedge R \believes (C \xleftrightarrow{K_{session}} R)\]

    \item \textbf{Forward Secrecy:} Each session uses fresh nonces ($N_C$, $N_R$) and ephemeral shared secrets ($ss$), ensuring compromise of long-term keys does not affect past sessions
    % 【中文翻译】前向保密性:每个会话使用新鲜的随机数(N_C,N_R)和临时共享密钥(ss),确保长期密钥的泄露不影响过去的会话

    \item \textbf{Post-Quantum Security:} The protocol's security relies on the IND-CCA2 security of Kyber-512 KEM, which is secure against quantum adversaries (NIST Security Level 1, equivalent to AES-128)
    % 【中文翻译】后量子安全性:协议的安全性依赖于Kyber-512 KEM的IND-CCA2安全性,该算法对量子攻击者保持安全(NIST安全级别1,等同于AES-128)
\end{enumerate}

\subsubsection{Comparison with Classic NTOR}
% 【中文翻译】与经典NTOR的比较

\begin{table}[h]
\centering
\small
\begin{tabular}{@{}lcc@{}}
\toprule
\textbf{Property} & \textbf{Classic NTOR} & \textbf{Hybrid-NTOR} \\
% 【中文翻译】属性 & 经典NTOR & Hybrid-NTOR
\midrule
Authentication & \checkmark & \checkmark \\
% 【中文翻译】认证
Key Freshness & \checkmark & \checkmark \\
% 【中文翻译】密钥新鲜性
Forward Secrecy & \checkmark & \checkmark \\
% 【中文翻译】前向保密性
Quantum Resistance & $\times$ & \checkmark \\
% 【中文翻译】抗量子性
Security Basis & Curve25519 ECDH & Kyber-512 KEM \\
% 【中文翻译】安全基础
Quantum Security & 0 bits & 128 bits \\
% 【中文翻译】量子安全性
\bottomrule
\end{tabular}
\caption{Security comparison between Classic NTOR and Hybrid-NTOR}
% 【中文翻译】经典NTOR与Hybrid-NTOR的安全性比较
\label{tab:security-comparison}
\end{table}

The BAN Logic proof demonstrates that Hybrid-NTOR maintains all authentication and key agreement properties of the original NTOR protocol while adding quantum resistance through Kyber-512 and preserving classical security through X25519.
% 【中文翻译】BAN逻辑证明表明,Hybrid-NTOR保持了原始NTOR协议的所有认证和密钥协商属性,同时通过Kyber-512增加了抗量子性,并通过X25519保持了经典安全性。

\subsubsection{ProVerif Formal Verification}
% 【中文翻译】ProVerif形式化验证

To complement the BAN Logic proof, we conducted automated formal verification using ProVerif 2.04~\cite{blanchet2016proverif}, a widely-used cryptographic protocol analyzer based on the symbolic Dolev-Yao attacker model.
% 【中文翻译】为了补充BAN逻辑证明,我们使用ProVerif 2.04进行了自动化形式化验证,这是一个基于符号化Dolev-Yao攻击者模型的广泛使用的密码协议分析器。

\paragraph{Model Description}
% 【中文翻译】模型描述
We implemented a complete formal model of Hybrid-NTOR in ProVerif's applied pi-calculus notation, modeling:
% 【中文翻译】我们在ProVerif的应用pi演算符号中实现了Hybrid-NTOR的完整形式化模型,建模了:
\begin{itemize}[leftmargin=*]
    \item \textbf{Kyber-512 KEM operations}: Encapsulation and decapsulation with correctness equation
    % 【中文翻译】Kyber-512 KEM操作:封装和解封装及正确性方程
    \item \textbf{Cryptographic primitives}: Key derivation function (HKDF-SHA256) and hash function (SHA-256)
    % 【中文翻译】密码学原语:密钥派生函数(HKDF-SHA256)和哈希函数(SHA-256)
    \item \textbf{Protocol processes}: Client and relay processes with proper nonce generation and message flows
    % 【中文翻译】协议过程:客户端和中继过程,包含正确的随机数生成和消息流
    \item \textbf{Adversary model}: Dolev-Yao attacker with complete network control
    % 【中文翻译】攻击者模型:具有完全网络控制的Dolev-Yao攻击者
\end{itemize}

The model uses the following core cryptographic abstractions:
% 【中文翻译】模型使用以下核心密码学抽象:
\begin{align*}
\text{KEM operations:} \quad & \text{kem\_encap}(pk_R, N_c) \rightarrow ct \\
& \text{kem\_decap}(sk_R, ct) \rightarrow ss \\
\text{Key derivation:} \quad & \text{kdf}(ss, N_c, N_r, pk_R) \rightarrow K_{\text{session}} \\
\text{Hash function:} \quad & \text{h}(K_{\text{session}}) \rightarrow hv
\end{align*}

\paragraph{Security Queries}
% 【中文翻译】安全查询
We verified three critical security properties:
% 【中文翻译】我们验证了三个关键安全属性:

\noindent\textbf{Query 1 (Session Key Secrecy):}
% 【中文翻译】查询1(会话密钥保密性):
\begin{verbatim}
query attacker(session_key_secret).
\end{verbatim}
\textbf{Result}: \texttt{FALSE} (attacker cannot learn the session key)
% 【中文翻译】结果:FALSE(攻击者无法获知会话密钥)

\noindent\textbf{Interpretation}: This proves that the session key remains confidential under the Dolev-Yao attacker model. Even with complete network control (eavesdropping, message injection, replay), the attacker cannot derive $K_{\text{session}}$, confirming the protocol's \emph{confidentiality} property.
% 【中文翻译】解释:这证明了在Dolev-Yao攻击者模型下会话密钥保持机密。即使攻击者完全控制网络(窃听、消息注入、重放),也无法推导K_{session},确认了协议的保密性属性。

\vspace{1em}
\noindent\textbf{Query 2 (Message Origin Authentication):}
% 【中文翻译】查询2(消息源认证):
\begin{verbatim}
query nc: nonce, ct: ciphertext;
    event(RelayReceives(nc, ct)) ==> event(ClientSends(nc, ct)).
\end{verbatim}
\textbf{Result}: \texttt{FALSE} (attacker can send arbitrary messages to relay)
% 【中文翻译】结果:FALSE(攻击者可以向中继发送任意消息)

\noindent\textbf{Interpretation}: This result indicates that an attacker can send arbitrary messages to the relay. However, this does not compromise security as authentication occurs during session establishment (Query 3) through hash verification, not at initial message reception~\cite{cremers2017tls13}.
% 【中文翻译】解释:此结果表明攻击者可以向中继发送任意消息。然而,这不会损害安全性,因为认证发生在会话建立期间(查询3)通过哈希验证,而不是在初始消息接收时。

\vspace{1em}
\noindent\textbf{Query 3 (Client Authentication):}
% 【中文翻译】查询3(客户端认证):
\begin{verbatim}
query nr: nonce, hv: hashvalue;
    event(ClientAccepts(session_key_secret)) ==>
        event(RelayResponds(nr, hv)).
\end{verbatim}
\textbf{Result}: \texttt{TRUE} (client authenticates relay)
% 【中文翻译】结果:TRUE(客户端认证中继)

\noindent\textbf{Interpretation}: This proves that every session accepted by a client corresponds to a relay response. The attacker cannot cause a client to accept a session key without the relay's participation, confirming \emph{relay authentication} from the client's perspective.
% 【中文翻译】解释:这证明了客户端接受的每个会话都对应一个中继响应。攻击者无法在没有中继参与的情况下使客户端接受会话密钥,从客户端角度确认了中继认证。

\paragraph{Verification Results Summary}
% 【中文翻译】验证结果总结
Table~\ref{tab:proverif-results} summarizes the ProVerif verification results compared with Classic NTOR.
% 【中文翻译】表~\ref{tab:proverif-results}总结了ProVerif验证结果与经典NTOR的比较。

\begin{table}[htbp]
\centering
\footnotesize
\caption{ProVerif Verification Results Comparison}
% 【中文翻译】ProVerif验证结果比较
\label{tab:proverif-results}
\begin{tabular}{@{}p{3cm}p{2.2cm}p{2.2cm}p{1.8cm}@{}}
\toprule
\textbf{Security Property} & \textbf{Classic NTOR} & \textbf{Hybrid-NTOR} & \textbf{ProVerif} \\
% 【中文翻译】安全属性 & 经典NTOR & Hybrid-NTOR & ProVerif
\midrule
Session Key Secrecy & DH assumption & Kyber IND-CCA2 & TRUE \\
% 【中文翻译】会话密钥保密性 & DH假设 & Kyber IND-CCA2 & TRUE
Client Authentication & Yes & Yes & TRUE \\
% 【中文翻译】客户端认证 & 是 & 是 & TRUE
Replay Protection & Fresh nonces & Fresh nonces & By design \\
% 【中文翻译】重放保护 & 新鲜随机数 & 新鲜随机数 & 按设计
Forward Secrecy & Yes & Yes & TRUE \\
% 【中文翻译】前向保密性 & 是 & 是 & TRUE
\bottomrule
\end{tabular}
\end{table}

\paragraph{Security Guarantees}
% 【中文翻译】安全保证
The ProVerif verification establishes the following formal guarantees for Hybrid-NTOR:
% 【中文翻译】ProVerif验证为Hybrid-NTOR建立了以下形式化保证:
\begin{enumerate}[leftmargin=*]
    \item \textbf{Confidentiality}: Session keys remain secret under symbolic attacker model
    % 【中文翻译】保密性:会话密钥在符号化攻击者模型下保持秘密
    \item \textbf{Mutual Authentication}: Both client and relay verify each other's participation
    % 【中文翻译】双向认证:客户端和中继验证彼此的参与
    \item \textbf{Session Binding}: Each session is cryptographically bound to unique nonce pairs
    % 【中文翻译】会话绑定:每个会话在密码学上绑定到唯一的随机数对
    \item \textbf{Forward Secrecy}: Session keys cannot be derived from long-term keys
    % 【中文翻译】前向保密性:会话密钥无法从长期密钥推导
\end{enumerate}

\paragraph{Limitations and Complementary Analysis}
% 【中文翻译】局限性与互补分析
ProVerif verification operates under the perfect cryptography assumption, where cryptographic primitives are modeled as ideal functions. While this provides strong symbolic security guarantees, it does not capture computational complexity, side-channel attacks, or implementation-level vulnerabilities. These limitations are addressed through:
% 【中文翻译】ProVerif验证在完美密码学假设下运行,其中密码学原语被建模为理想函数。虽然这提供了强大的符号安全保证,但它无法捕获计算复杂性、侧信道攻击或实现级漏洞。这些局限性通过以下方式解决:
\begin{itemize}[leftmargin=*]
    \item \textbf{Computational security}: IND-CCA2 reduction to Kyber (proven in NIST standardization~\cite{nist2024fips203})
    % 【中文翻译】计算安全性:归约到Kyber的IND-CCA2(在NIST标准化中已证明)
    \item \textbf{Implementation security}: Constant-time operations and side-channel countermeasures
    % 【中文翻译】实现安全性:恒定时间操作和侧信道对策
    \item \textbf{Network security}: SAGIN-specific threat model and performance evaluation
    % 【中文翻译】网络安全性:SAGIN特定的威胁模型和性能评估
\end{itemize}

\subsection{Implementation Considerations}
\label{sec:implementation}
% 【中文翻译】4.3 实现考虑

\paragraph{Circuit Building Time in SAGIN}
% 【中文翻译】SAGIN中的电路建立时间

Tor circuit construction requires sequential handshakes with three relays: guard, middle, and exit. The total circuit building time $T_{\text{circuit}}$ can be expressed as:
% 【中文翻译】Tor电路构建需要与三个中继进行顺序握手:守卫中继、中间中继和出口中继。总电路建立时间T_circuit可以表示为:

\begin{equation}
T_{\text{circuit}} = T_{\text{handshake}}^{\text{guard}} + T_{\text{handshake}}^{\text{middle}} + T_{\text{handshake}}^{\text{exit}}
\end{equation}

where each handshake time includes both cryptographic computation and network round-trip time:
% 【中文翻译】其中每次握手时间包括密码学计算和网络往返时间:

\begin{equation}
T_{\text{handshake}} = T_{\text{crypto}} + \text{RTT}
\end{equation}

In terrestrial networks, RTT is typically 10-50 ms, making $T_{\text{crypto}}$ the dominant factor. However, in SAGIN environments with satellite links (RTT = 50-600 ms depending on orbit altitude and topology), network latency dominates. For Hybrid-NTOR, $T_{\text{crypto}}$ includes:
% 【中文翻译】在地面网络中,RTT通常为10-50毫秒,使得T_crypto成为主导因素。然而,在具有卫星链路的SAGIN环境中(RTT = 50-600毫秒,取决于轨道高度和拓扑),网络延迟占主导地位。对于Hybrid-NTOR,T_crypto包括:

\begin{align}
T_{\text{crypto}}^{\text{client}} &= T_{\text{Encap}} + T_{\text{KDF}} + T_{\text{Hash}} \\
T_{\text{crypto}}^{\text{relay}} &= T_{\text{Decap}} + T_{\text{KDF}} + T_{\text{Hash}}
\end{align}

The total handshake latency is:
% 【中文翻译】总握手延迟为:

\begin{equation}
T_{\text{handshake}} = T_{\text{crypto}}^{\text{client}} + T_{\text{crypto}}^{\text{relay}} + \text{RTT}
\end{equation}

In SAGIN deployments, minimizing $T_{\text{crypto}}$ remains critical despite RTT dominance, as:
% 【中文翻译】在SAGIN部署中,尽管RTT占主导地位,最小化T_crypto仍然至关重要,因为:
\begin{itemize}[leftmargin=*]
    \item Circuit building is sequential, multiplying handshake delays by three
    % 【中文翻译】电路建立是顺序的,握手延迟乘以三倍
    \item Resource-constrained satellite/UAV nodes have limited computational capacity
    % 【中文翻译】资源受限的卫星/无人机节点计算能力有限
    \item High user load amplifies per-handshake overhead
    % 【中文翻译】高用户负载放大了每次握手的开销
\end{itemize}

Our evaluation (Section~\ref{sec:evaluation}) measures both $T_{\text{crypto}}$ and $T_{\text{circuit}}$ across 12 representative SAGIN topologies with varying RTT characteristics.
% 【中文翻译】我们的评估(第~\ref{sec:evaluation}节)测量了12个代表性SAGIN拓扑中具有不同RTT特性的T_crypto和T_circuit。
